\documentclass[fontset=fandol,UTF8,fleqn]{beamer}
% \usepackage[english]{babel}
%\PassOptionsToPackage{quiet}{fontspec}
\usepackage[UTF8]{ctex}
%\usepackage{fontspec}

%\usepackage[useregional]{datetime2}
%\usepackage{mathtools,amsthm,amssymb,bm,amsmath,mathrsfs}
\usepackage{amssymb,amsfonts,amsmath,amsthm,mathrsfs,mathptmx} %数学常用宏包
\usepackage{amsfonts}
 \usepackage{graphicx}
% \usepackage[backend=biber,style=authoryear]{biblatex}
% \addbibresource{biblio.bib}

\newcommand{\projdim}[1]{\mathbb{P}^#1}
\newcommand{\esperanza}[1]{\mathbb{E}(#1)}
\newcommand{\proba}[1]{\mathbb{P}(#1)}
\newcommand{\gavilla}[2]{\mathcal{O}_{#1}(#2)}

 \newcommand{\citap}[1]{\textcolor{blue}{(\cite{#1})}}

% diagramas
\usepackage{tikz, tikz-cd}
\usepackage{pgf}
\usetikzlibrary{cd, babel}  % evita errores en tikz
\usetikzlibrary{graphs}
\usetikzlibrary{graphs.standard}
\usetikzlibrary{arrows,automata}


% vertical separator macro
\newcommand{\vsep}{
  \column{0.0\textwidth}
    \begin{tikzpicture}
      \draw[very thick,black!10] (0,0) -- (0,7.3);
    \end{tikzpicture}
}

% More space between lines in align
\setlength{\mathindent}{0pt}

% % Beamer theme
 \usetheme{ZMBZFMK}
\usefonttheme[onlysmall]{structurebold}
\mode<presentation>
\setbeamercovered{transparent=10}

% align spacing
\setlength{\jot}{0pt}

 \setCJKmainfont[ItalicFont={AR PL UKai CN}]{AR PL UMing CN}
\setCJKmainfont[ItalicFont={AR PL UKai CN}]{SimSun}
 \setCJKsansfont{SimSun}
\setCJKmonofont{FangSong}

% \AtBeginSection[]
% {
%   \begin{frame}
%     \frametitle{Table of Contents}
%     \tableofcontents[currentsection]
%   \end{frame}
% }
% \AtBeginSubsection[]
% {
%   \begin{frame}
%     \frametitle{Table of Contents}
%     \tableofcontents[currentsubsection]
%   \end{frame}
% }


% Only the first Slide
\title{{算法设计与分析}}
\author{MHT}
\institute[Northeastern University at Qinhuangdao]{ }
\date{\today}


\begin{document}

\begin{frame}
  \titlepage
\end{frame}

\section{Divide and Conquer}

\begin{frame}{第2章 递归与分治策略}
\begin{itemize}[<+-|alert@+>]
\item 理解递归的概念
\item 掌握分治算法设计策略
\item 二分查找技术; 大整数乘法; 矩阵乘法; 合并排序;  快速排序 
\end{itemize}
\end{frame}

\begin{frame}{分治策略概述}
  \begin{exampleblock}{分治策略}
    将一个难以直接解决的大问题, 分割成一些规模较小的相同子问题, 各个击
    破, ``\textbf{分而治之}'' 
  \end{exampleblock}

 \begin{itemize}[<+-|alert@+>]
  \item 将要求解的问题划分成~$k$~个小问题
  \item 然后对这~$k$~个子问题\underline{递归求解}
  \item 合并子问题的解
\end{itemize}
\end{frame}

\begin{frame}{分治策略应用的典型情况}
\begin{figure}
  \centering
  \includegraphics[width=.8\textwidth]{./pictures/divide-and-conquer.png}%
%  \caption{分治技术的典型情况}
  \label{fig:dcm}
\end{figure}
\end{frame}

\begin{frame}{2.1 递归的概念}
\begin{definition}[递归函数]
  用函数自身给出定义的函数称为递归函数.  
\end{definition}
\begin{definition}[递归算法]
直接或间接地调用自身的算法称为递归算法.  
\end{definition}
\end{frame}

\begin{frame}{递归定义条件}
对于函数 $f(n)$ 的递归定义, 必须满足如下条件: 
\begin{itemize}[<+-|alert@+>]
\item \textbf{基本部分:}  其中对于 $n$ 的一个或多个值, $f(n)$ 必须是直接定义的(即非递归)  
\item \textbf{递归部分:} 右侧所出现的所有 $f$ 的参数都必须有一个比 $n$ 小, 以重复运用递归部分来改变右侧出现的 $f$ , 直至出现 $f$ 的基本部分 
\end{itemize}
\end{frame}

\begin{frame}{例2-1 阶乘函数}
  阶乘函数$f(n)$可递归地定义为
  \begin{exampleblock}{}
\begin{displaymath}
\boldsymbol{  f(n)}=\left\{\begin{array}{ll}
1 & ,n=0 \hspace{5cm}  (1) \\
n\times f(n-1) & ,n >0 \hspace{5cm} (2) 
\end{array}\right. 
\end{displaymath}
\end{exampleblock}
\begin{itemize}[<+-|alert@+>]
\item 阶乘函数的自变量~$n$~的定义域是非负整数 
\item 定义式的第(1)式给出了函数的初始值, 是非递归定义 
\item 定义式的第(2)式左右两边都引用了阶乘函数记号, 是一个递归定义式 
\end{itemize}
\end{frame}

\begin{frame}{算法描述}
  \begin{exampleblock}{Algorithm. $F(n)$.}
\qquad \textbf{if} $n = 0$: \\
\qquad\qquad \textbf{return} 1 \\
\qquad \textbf{else:} \\
\qquad\qquad \textbf{return} $F(n-1)*n$
\end{exampleblock}
\end{frame}

\begin{frame}{阶乘函数算法计算复杂性分析}
\begin{itemize}[<+-|alert@+>]
\item 算法输入规模为 $n$  
\item 算法基本操作是乘法  
\item 算法基本操作执行次数记作 $M(n)$ 
\end{itemize}
\end{frame}

\begin{frame}{计算 $M(n)$}
\begin{itemize}[<+-|alert@+>]
\item 当 $n>0$ 时, $F(n)=F(n-1)* n$  
\item 计算 $F(n)$ 时, 用到的乘法数量 $M(n)$ 需要满足公式  
  \begin{displaymath}
    M(n) = M(n-1) + 1, \qquad n > 0  
  \end{displaymath}
\item $M(n)=M(n-1)+1$ 定义了需要求值的 $M(n)$ 的序列, 该式为一递归式 
\item 为了解递归式, 需要一个初始条件说明序列的初始值  
\end{itemize}
\end{frame}

\begin{frame}{分析递归式 $M(n)=M(n-1)+1$}
  \begin{exampleblock}{Algorithm. $Fac(n)$.}
    \qquad \textbf{if} $n = 0$: \\
    \quad\qquad \textbf{return} 1 \\
\qquad \textbf{else:} \\
\qquad\qquad \textbf{return} $F(n-1)*n$
\end{exampleblock}
\begin{itemize}[<+-|alert@+>]
\item 调用在 $n=0$ 时结束,算法处理的 $n$ 的最小值是 $0$  
\item 当 $n=0$ 时, 该算法不执行乘法操作 
\end{itemize}
\end{frame}

\begin{frame}{算法复杂性递归关系和初始条件}
  \begin{exampleblock}{M(n).}
    \begin{displaymath}
  \left\{ \begin{array}{lll}
  M(n) & = & M(n-1)+1, \qquad n > 0 \\
M(0) & = & 0, \qquad\ \qquad \quad \qquad n = 0
\end{array}\right.
\end{displaymath}
\end{exampleblock}
\end{frame}

\begin{frame}{Fun Time}
  \begin{exampleblock}{M(n)=n.}
    \begin{eqnarray*}
  M(n) & = & M(n-1)+ 1     \\
& = & [M(n-2)+1]+1  =  M(n-2)+2    \\
& = & [M(n-3)+1]+2 = M(n-3)+3   \\
\ldots   \\
\Rightarrow M(n) & = & \underline{M(n-i)+i}  \\
\Rightarrow M(n) & = & M(n-n)+n=n.  
    \end{eqnarray*}
  \end{exampleblock}
\end{frame}

\begin{frame}{递归算法复杂性分析方案}
\begin{itemize}[<+-|alert@+>]
\item[(1)] 决定用哪个参数表示输入规模度量  
\item[(2)] 找出算法的基本操作  
\item[(3)] 检查基本操作的执行次数是否只依赖于输入规模  
\item[(4)] 建立算法基本操作执行次数的递归式及相应的初始条件   
\item[(5)] 解递归式, 或者确定它的渐近函数  
\end{itemize}
\end{frame}

\begin{frame}{例2-2 Fibonacci~数列}
  无穷数列 1, 1, 2, 3, 5, 8, 13, \ldots,  称为~Fibonacci~数列:
  \begin{exampleblock}{}
     \begin{displaymath}
   F(n)=\left\{\begin{array}{ll}
1 & ,n=0\\
1 & ,n=1\\
F(n-1)+F(n-2) & ,n>1
\end{array}\right. 
\end{displaymath}
\end{exampleblock}
\end{frame}

\begin{frame}{Fibonacci 数列计算算法}
  \begin{exampleblock}{Algorithm.$Fib(n)$}
\qquad    \textbf{if} $n \leq 1$:\\
\qquad \qquad \textbf{return} 1\\
\qquad    \textbf{return} $Fib(n-2) + Fib(n-1)$
\end{exampleblock}
\end{frame}

\begin{frame}{阶乘与 Fibonacci 的非递归定义}
  上述两例中的函数也可用如下非递归方式定义:
  \begin{exampleblock}{阶乘}
    \begin{displaymath}
  n!=1\times 2\times 3\times \cdots \times (n-1)\times n
\end{displaymath}
\end{exampleblock}
\begin{exampleblock}{ Fibonacci}
  \begin{displaymath}
  F(n)=\frac{1}{\sqrt{5}}\Bigg[\bigg(\frac{1+\sqrt{5}}{2}\bigg)^{n+1}-\bigg(\frac{1-\sqrt{5}}{2}\bigg)^{n+1}\Bigg]
\end{displaymath}
\end{exampleblock}
\end{frame}

\begin{frame}{例2-3 Ackerman~函数}
%当一个函数及它的一个变量是由函数自身定义时, 称这个函数是双递归函数. 
Ackerman~函数 $A(n, m)$ 包含两个独立整形变量$m \geq 0$, $n\geq 0$, 定
义如下:
\begin{exampleblock}{Ackerman Function}  
\begin{eqnarray*}
  \left\{\begin{array}{lll}
A(1, 0) & =2 & \\
A(0, m)& =1 &  m \geq 0\\
A(n, 0) & = n + 2 & n \geq 2\\
A(n, m)& =A(A(n-1, m), m-1) & n, m \geq 1
\end{array}\right. 
\end{eqnarray*}
\end{exampleblock}
\end{frame}

\begin{frame}{}
  \begin{exampleblock}{}
    $A(n,  m)$~的自变量~$m$~的每一个值都定义了一个单变量函数,  例如:  

$m=0$~时,    $A(n, 0)=n+2$; 

 $m=1$~时,   \\
$A(1, 1)=A(\underline{A(0,1)},0)=A(\underline{1},0)=2$, 则  
 \begin{eqnarray*}
      A(n, 1) & = & A(\underline{A(n-1, 1)}, 0)  \\
& = & \underline{A(n-1, 1)}+2  \\  
& = & 2*n  
 \end{eqnarray*}
\end{exampleblock}
\end{frame}

\begin{frame}{Fun Time}
  请根据
    \begin{displaymath}
  A(n, 1)  =  A(n-1, 1)+2
\end{displaymath}
推导出$A(n,1)=2*n$ [提示:反向替换法].
\end{frame}

\begin{frame}{$A(n, 1)=2*n$ 的推导}
  \begin{exampleblock}{$A(n, 1)=2*n$ 的推导}
\begin{eqnarray*}
  A(n, 1) & = & \underline{A(n-1, 1)}+2   \\
 &  = & \underline{[A(n-2, 1)+2]}+2   \\  
&  = & [A(n-3, 1)+2]+2* 2   \\
%\vdots & = & \vdots  \\
& = & A(n-i, 1)+2* i  \\
%\vdots & = & \vdots  \\
A(n, 1) & = & A(n-(n-1), 1)+(n-1)* 2   \\
& = & A(1,1)+(n-1)* 2   \\
& = & 2*n  
\end{eqnarray*}
\end{exampleblock}
\end{frame}

\begin{frame}{Ackerman~函数}
  \begin{exampleblock}{}
$m=2$~时,  $A(n, 2)=A(A(n-1, 2), 1)=2A(n-1, 2)$,  和~$A(1, 2)=A(A(0, 2), 1)=A(1, 1)=2$,  $A(n, 2)= 2^n$; 

$m=3$~时,  类似的可以推出~$A(n, 3)=2^{2^{\cdot^{\cdot^{\cdot^2}}}} $; 

$m=4$~时,  $A(n, 4)$~的增长速度非常快, 没有适当的数学式子来表示这一函数.  
\end{exampleblock}
\end{frame}

\begin{frame}{补充 \underline{Ackerman} 函数}
  \begin{exampleblock}{Acerman Function}
    \begin{displaymath}
A(m, n)=\left \{ \begin{array}{ll}
n + 1 & if\ m = 0 \\
A(m-1, 1) & if\ m > 0\ and\ n = 0 \\
A(m-1, A(m, n-1)) & if\ m > 0\ and\ n > 0
\end{array}\right.
\end{displaymath}
\end{exampleblock}
\end{frame}

\begin{frame}{Ackermann函数算法}
  \begin{exampleblock}{Algorithm.$Ack(m, n)$}
\qquad \textbf{if} $m == 0$:\\
\qquad \qquad     \textbf{return} $n + 1$\\
\qquad \textbf{if} $m > 0\ and\ n == 0$:\\
\qquad \qquad     \textbf{return} $Ack(m-1, 1)$\\
\qquad \textbf{if} $m > 0\ and\ n > 0$:\\
\qquad \qquad     \textbf{return} $Ack(m-1, Ack(m, n-1))$
\end{exampleblock}
\end{frame}

\begin{frame}{例2-4 排列问题}
  \begin{exampleblock}{排列问题}
    对~$R=\{r_1, r_2, \ldots,r_n\}$, ~$n$~元素排列. 
  \end{exampleblock}
  \begin{itemize}[<+-|alert@+>]
\item $R_i=R-\{r_i\}$  
\item 集合~$X$的全排列记为~$\mathrm{Perm}(X)$  
\item $(r_i)\mathrm{Perm}(X)$~表示在全排列~$\mathrm{Perm}(X)$ 
的每一个排列前加上前缀得到的排列   
\end{itemize}
\end{frame}

\begin{frame}{$R$全排列递归定义}
\begin{itemize}[<+-|alert@+>]
\item 当~$n=1$~时,  $\mathrm{Perm}(R)=(r)$,  其中~$r$~是集合~$R$~中唯一的元素; 
\item 当~$n>1$~时, $\mathrm{Perm}(R)$~由~ $(r_1)\mathrm{Perm}(R_1)$
 , $\ldots$, $(r_n)\mathrm{Perm}(R_n)$~构成. 
\end{itemize}
\end{frame}

\begin{frame}{排列基本计算思想}
  $R=\{1, 2, 3, 4\}$, 则 $R_1=R-r_1=\{2, 3, 4\}$,
  \begin{exampleblock}{Idea of Permutation.}
    \begin{eqnarray*}
  (r_1)\mathrm{Perm}(R_1) & = & \mathbf{1} \mathrm{Perm}(R_1)   \\
& = & \mathbf{1}\quad  2\quad 3 \quad 4\\ 
& & \mathbf{1}\quad  2\quad 4 \quad 3   \\ 
& & \mathbf{1}\quad  3\quad 2 \quad 4  \\ 
& & \mathbf{1}\quad  3\quad 4 \quad 2   \\
& & \mathbf{1}\quad  4\quad  {3 \quad 2}  \\
& & \mathbf{1}\quad  4\quad {2 \quad 3}  
\end{eqnarray*}
\end{exampleblock}
\end{frame}

\begin{frame}{排列问题求解算法}
  \begin{exampleblock}{Algorithm.$Perm(A, k, m)$}
\qquad    \textbf{if} $(k == m) $\\
 \qquad \qquad  \textbf{for} $x\ in\ A$:  \qquad \textbf{print} $x$\\
\qquad \textbf{else}\\
\qquad \qquad $i = k$\\
\qquad \qquad \textbf{while} $i \leq m$\\
\qquad \qquad \qquad $A[k], A[i] = A[i], A[k]$\\
\qquad \qquad \qquad $Perm(A, k+1, m)$\\
\qquad \qquad \qquad $ A[k], A[i] = A[i], A[k]$\\
\qquad \qquad \qquad $i = i + 1$
\end{exampleblock}
\end{frame}

\begin{frame}{例2-5 整数划分问题}
\begin{itemize}[<+-|alert@+>]
\item 将正整数~$n$~表示成一系列正整数之和: 
\begin{displaymath}
  n=n_1+n_2+\ldots +n_k, (n_1\geq n_2 \geq \ldots \geq n_k \geq 1,  k\geq 1) 
\end{displaymath}
\item 正整数~$n$~的这种表示称为\textbf{正整数~$n$~的划分} 
\item 正整数 $n$ 的不同的划分个数称为 $n$ 的划分数, 记作~$p(n)$   
\item \textbf{整数划分问题}: 求正整数~$n$~的不同划分个数   
\end{itemize}
\end{frame}

\begin{frame}{Fun Time}
  正整数 $5$ 的不同划分个数.
  \begin{exampleblock}{}
    \begin{displaymath}
  n=n_1+n_2+\ldots +n_k, (n_1\geq n_2 \geq \ldots \geq n_k \geq 1,  k\geq 1) 
\end{displaymath}
\end{exampleblock}
\end{frame}

\begin{frame}{正整数5的划分}
  正整数 $5$ 有如下 $7$ 种不同的划分:  
  \begin{exampleblock}{}
    \begin{eqnarray*}
&   & 5;   \\
 &   & 4+1;   \\
& &   3+2,  3+1+1;   \\
 & & 2+2+1,  2+1+1+1;  \\
  & & 1+1+1+1+1.   
 \end{eqnarray*}
\end{exampleblock}
\end{frame}

\begin{frame}{整数划分算法思想}
\begin{itemize}[<+-|alert@+>]
\item 将最大加数~$n_1$~不大于~$m$~的划分个数记作~$q(n, m)$ 
\item 建立~$q(n, m)$~如下递归关系   
\end{itemize}
\end{frame}

\begin{frame}{$q(n,m)$递归公式}
\begin{itemize}[<+-|alert@+>]
  \item[(1)] $n\geq 1$: $\boldsymbol{q(n,1)=1}$,  
%当最大加数 $n_1$ 不大于 1 时,任何正整数 $n$ 只有一种划分形式,即 
$n=\overbrace{1+1+\cdots +1}$. 
  \item[(2)] $m>n$: $\boldsymbol{q(n,m)=q(n,n)}$,
%; 最大加数 $n_1$ 实际上不能大于 $n$.  因此,
 $q(1,m)=1$. 
\item[(3)] $m=n$: $\boldsymbol{q(n,m)=1+q(n,n-1)}$, 正整数 $n$ 的划分由 $n_1=n$
  的划分和 $n_1\leq n-1$ 的划分组成. 
  \item[(4)] $n>m>1$: $q(n,m)=q(n,m-1)+q(n-m,m)$; 正整数 $n$ 的最大加数 $n_1
$ 不大于 $m$ 的划分由 $n_1=m$ 的划分和 $n_1\leq m-1$ 的划分组成. 
\end{itemize}
\end{frame}

\begin{frame}{}
  以上的关系给出了计算~$q(n, m)$~的递归式如下:  
  \begin{exampleblock}{}
    \begin{displaymath}
\boldsymbol{ q(n, m)=}\left\{\begin{array}{lll}
1 & $n=1 or\   m=1$   \\
q(n, n) & n<m   \\
1+q(n, n-1) & n=m   \\
q(n, m-1)+q(n-m, m) & n>m>1  
\end{array} \right. 
\end{displaymath}
\end{exampleblock}
\end{frame}

\begin{frame}{Hanoi塔问题}
  \begin{exampleblock}{Hanoi塔问题}
     我们有 $n$ 个不同大小的盘子和3根柱子, 开始时所有盘子都按照大小顺序套在第一根柱子上, 最大的盘子在底部, 最小的盘子在顶部.  

目的是把所有的盘子都移到第3根柱子上, 必要的时候可以借助第二根柱子. 每次只能移动
一个盘子, 但是不能把较大的盘子放在较小的盘子上面.  
\end{exampleblock}
\end{frame}

\begin{frame}{Hanoi 问题递归算法思想}
\end{frame}

\begin{frame}{算法复杂性分析}
\begin{itemize}[<+-|alert@+>]
\item 选择盘子的数量 $n$ 作为输入规模度量 
\item 选择盘子的移动作为算法的基本操作  
\item 移动次数为 $M(n)$  
\end{itemize}
\end{frame}

\begin{frame}{建立递归关系式}
  \begin{exampleblock}{}
    \begin{displaymath}
  \left \{ \begin{array}{lll}
  M(n) & = & 2M(n-1)+1, \qquad n > 1 \\
M(1) & = & 1, \qquad \qquad \qquad \qquad  n =1
\end{array} \right.
\end{displaymath}
\end{exampleblock}
\end{frame}

\begin{frame}{Fun Time}
  \begin{exampleblock}{ Hanoi 塔递归关系式求解}
    \begin{displaymath}
  \left \{ \begin{array}{lll}
  M(n) & = & 2M(n-1)+1, \qquad n > 1 \\
M(1) & = & 1, \qquad \qquad \qquad \qquad  n =1
\end{array} \right.
\end{displaymath}
\end{exampleblock}
\end{frame}

\begin{frame}{Fun Time}
  \begin{exampleblock}{}
    \begin{eqnarray*}
  M(n) & = & 2M(n-1)+1   \\
& = & 2[2M(n-2)+1]+1 = 2^2M(n-2)+2+1  \\
& = & 2^2[2M(n-3)+1]+2+1   \\
& = & 2^3M(n-3)+2^2+2+1  \\
\ldots   \\
\Rightarrow M(n) & = & 2^i(M(n-i)+2^i-1   \\
\Rightarrow M(n) & = & 2^{n-1}M(n-(n-1))+2^{n-1}-1   \\
& = & 2^n -1   
    \end{eqnarray*}
  \end{exampleblock}
\end{frame}

\begin{frame}{Hanoi问题求解算法}
  \begin{exampleblock}{Algorithm. $Hanoi(n, source, helper, target)$}
\textbf{if} $ n > 0$\\
%\qquad \# move tower of size $n-1$ to helper:\\
\qquad $Hanoi(n - 1, source, target, helper)$\\
%\qquad\# move disk from source peg to target peg\\
\qquad \textbf{if} source\\
\qquad \qquad target.append(source.pop())\\
%\qquad \# move tower of size $n-1$ from helper to target\\
\qquad $ hanoi(n - 1, helper, source, target)$
\end{exampleblock}
\end{frame}

\begin{frame}{Tips}
\begin{center}
\huge{\textbf{应谨慎使用递归算法, 它们的简洁可能会掩盖它们的低效性.}}
\end{center}
\end{frame}

\begin{frame}{递归小结}
\begin{itemize}[<+-|alert@+>]
\item 结构清晰, 可读性强, 而且容易用数学归纳法来证明算法的正确性, 为设计算法、调试程序带来很大方便   
\item 递归算法的运行效率较低,  无论是耗费的计算时间还是占用的存储空间都比非递归算法要多   
\item 在递归算法中消除递归调用,  使其转化为非递归算法   
\end{itemize}
\end{frame}

\begin{frame}{2.2 分治法的基本思想}
分治法解决问题一般特征: 
\begin{itemize}[<+-|alert@+>]
\item[(1)]问题的规模缩小到一定程度就可以容易地解决 
\item[(2)] 问题可分解为若干规模较小的相同问题 
\item[(3)] 利用子问题的解可以合并为该问题的解 
\item[(4)] \underline{问题所分解出的子问题之间不包含公共子问题}   
\end{itemize}
\end{frame}

\begin{frame}{分治算法步骤}
\begin{itemize}[<+-|alert@+>]
\item 分(Divide): 将原始问题分解为较小的相同子问题 
\item 治(Conquer): 递归解决这些子问题 
\item 合并(Combine): 将较小子问题的解合并获得较大问题的解, 直到获得原始问题的解  
\end{itemize}
\end{frame}

\begin{frame}{Ceilings 和 Floors 函数}
\begin{itemize}[<+-|alert@+>]
\item 对于任意实数 $x$, 取底函数(Floor) $\lfloor x \rfloor$ 表示不大于(小于或等于) $x$ 的最大整数, 取顶函数(Ceiling) $\lceil x \rceil$ 表示不小于(大于或等于)
  $x$ 的最小整数, 对于任意实数 $x$,
  \begin{eqnarray*}
    \hspace{3cm}
\boldsymbol{  x - 1 < \lfloor x \rfloor \leq  x \leq \lceil  x \rceil  < x +1}.
\end{eqnarray*}
\item 对于任意整数 $n$,
  \begin{eqnarray*}
        \hspace{3cm}
\boldsymbol{  \lceil n/2 \rceil  + \lfloor n/2 \rfloor = n}.
\end{eqnarray*}
\end{itemize}
\end{frame}

\begin{frame}{分治法求和示例}
\begin{itemize}[<+-|alert@+>]
\item 问题: 计算 $n$ 个数字 $a_0,\ldots,a_{n-1}$ 的和. 
\item 分\&治: 分别计算前 $\lfloor n/2 \rfloor$ 个数字的和以及计算后 $\lceil n/2 \rceil$ 个数字的和. 
\item 合并: 将上述两个和相加, 得出原始问题答案: 
  \begin{eqnarray*}
 a_0+\cdots a_{n-1} = (a_0+\cdots a_{\lfloor n/2 \rfloor -1})+(a_{\lceil n/2 \rceil-1}+\cdots  a_{n-1}) 
  \end{eqnarray*}
\end{itemize}
\end{frame}

\begin{frame}{分治算法时间复杂性}
\begin{itemize}[<+-|alert@+>]
\item 使用递归思想解决问题  
\item 主要技巧是问题分解、子问题求解、子问题解合并  
\item 递归属性使其时间复杂性可表示为\underline{递归表达式}  
\end{itemize}
\end{frame}

\begin{frame}{分治算法时间复杂性公式}
  \begin{exampleblock}{Time Complexity}
    \begin{displaymath}
\boldsymbol{T(n)= aT(\lceil n/b \rceil)+f(n)}
\end{displaymath}
\end{exampleblock}
\begin{itemize}[<+-|alert@+>]
\item $a\geq 1$: 每次分解时产生的子问题的个数 
\item $b>1$: 每次分解时问题大小下降的因子 
%\item $O(n^d)$ 是将 $a$ 个大小为 $n/b$ 的子问题合并的成本. 
\item $f(n)$: 表示分解问题和合并子问题解的代价  
\end{itemize}
\end{frame}

\begin{frame}{主(Master)定理}
\begin{theorem}[主定理]给定常数 $a\geq 1$,  $b>1$,  $d\geq 0$,  假设 $n=b^k (k=1,2,\ldots$), 如果 $f(n) =
  \Theta(n^d)$, 即 

$T(n)=aT(\lceil n/b \rceil)+\Theta(n^d)$,  则
  \begin{eqnarray*}
    T(n) = \left\{
        \begin{array}{ll}
          \Theta(n^d) &\qquad  \mathrm{if}\ a < b^d   \\
\Theta(n^d\log n) &\qquad  \mathrm{if} \ a = b^d   \\
\Theta(n^{\log_ba}) &\qquad  \mathrm{if}\ a > b^d  
        \end{array}
\right. 
  \end{eqnarray*}
\end{theorem}
 (对于 $O$ 和 $\Omega$ 符号来说类似结论成立)   
\end{frame}

\begin{frame}{Fun Time}
  假设输入规模为 $n=2^k$, 采用主定理估算分治求和算法加法运算次数 $T(n)$.
  \begin{exampleblock}{Time Complexity}
      \begin{eqnarray*}
 a_0+\cdots a_{n-1} = (a_0+\cdots a_{\lfloor n/2 \rfloor -1})+(a_{\lceil n/2 \rceil}+\cdots  a_{n-1})
      \end{eqnarray*}
    \end{exampleblock}
    \begin{exampleblock}{Master}
          \begin{eqnarray*}
    T(n) = \left\{
        \begin{array}{ll}
          \Theta(n^d) &\qquad  if\ a < b^d  \\
\Theta(n^d\log n) &\qquad  if \ a = b^d  \\
\Theta(n^{\log_ba}) &\qquad  if\ a > b^d
        \end{array}
\right. 
          \end{eqnarray*}
        \end{exampleblock}
      \end{frame}

\begin{frame}{Fun Time}
\begin{itemize}[<+-|alert@+>]
\item 输入规模为 $n=2^k$ 时, 加法运算次数 $T(n)$ 可用如下递推公式表示:
  \begin{eqnarray*}
    T(n)=2T(n/2)+O(1)     
  \end{eqnarray*}

\item 其中, $a=2$, $b=2$, $d=0$. 因为 $a>b^d$, 有 
  \begin{eqnarray*}
    T(n)\in \Theta(n^{\log_ba})=\Theta(n^{\log_22})=\Theta(n). 
  \end{eqnarray*}
\end{itemize}
\end{frame}

\begin{frame}{2.3 二分搜索技术}
  \begin{exampleblock}{查找问题}
给定按升序排序的~$n$~个元素~$A[0..n-1]$, 在~$n$~个元素中找出一特定元
素~$x$.
\end{exampleblock}
\end{frame}

\begin{frame}{分治法解决问题一般特征} 
\begin{itemize}[<+-|alert@+>]
\item[(1)]问题的规模缩小到一定程度就可以容易地解决
\item[(2)] 问题可分解为若干规模较小的相同子问题
\item[(3)] 利用子问题的解可以合并为该问题的解
\item[(4)] {问题所分解出的子问题之间不包含公共子问题} 
\end{itemize}
\end{frame}

\begin{frame}{2. 查找问题满足分治策略的条件判定}
\begin{itemize}[<+-|alert@+>]
\item[(1)] 如果 $n=1$, 则只要比较这个元素和 $x$ 就可以确定 $x$ 是否存在(条件1成立).
\item[(2)] 比较 $x$ 和 $A$ 的中间元素 $A[m]$, 若 $x=A[m]$, 则 $x$ 在 $A$ 中的位置是 $m$; 若 $x<A[m]$, 则在 $A[0..m-1]$ 中查找 $x$; 若 $x>A[m]$, 则在 $A[m+1..n]$ 中查找 $x$(条件2,3成立).
\item[(3)] 问题分解出的子问题相互独立(条件4成立).
\end{itemize}
\end{frame}

\begin{frame}{3. 算法思想} 
二分搜索通过比较查找键 $x$ 和数组中间元素$A[m]$ 来完成\underline{有序数组}元素查找工作 
% 如果它们相等, 算法结束. 否则, 如果 $x<A[m]$, 就对数组的前半部分执行该操作, 如果 $x>A[m]$, 则对数组的后半部分执行该操作. 
\begin{exampleblock}{}
  \begin{eqnarray*}
\underbrace{A[0]\cdots A[m-1]}_{ if\ x<A[m]} & \boldsymbol{A[m]} & \underbrace{A[m+1]\cdots A[n-1]}_{if\ x>A[m]}  
\end{eqnarray*}
\end{exampleblock}
\end{frame}

\begin{frame}{4. 二分搜索算法}
\textbf{binary\_search}$(A, k, l, r)$:\\
\qquad    \textbf{if} $l > r$: \textbf{return} $-1$\\
\qquad    $m = (l+r)/2$\\
\qquad      \textbf{if} $k == A[m]$: \textbf{return} $m$ \\
\qquad      \textbf{elif} $k < A[m]$: \\
\qquad \qquad \textbf{return} \textbf{binary\_search}$(A, k, l, m-1)$ \\
\qquad     \textbf{else}: \\
\qquad \textbf{return} \textbf{binary\_search}$(A, k, m+1, r)$
\end{frame}

\begin{frame}{5. 二分搜索算法时间复杂性分析}
  \begin{exampleblock}{Time Complexity}
    \begin{eqnarray*}
  T(n) & = & aT(n/b)+f(n) \\
   & = & T(n/2)+O(1)  
 \end{eqnarray*}
其中, $a = 1, b=2, n^d=1$, 即 $d = 0$, 由于有 $a  = b^d$ 成立, 根据主定理, 
\begin{eqnarray*}
   \Rightarrow    T(n) & = &   O(n^d \log n)  \\
                       & = & O(n^0 {\log n}) \\    
                       &  = & O(\log n) 
\end{eqnarray*}
\end{exampleblock}
\end{frame}

\begin{frame}{6. 二分搜索算法实例}
  $x=70$, 对下面的数组应用二分搜索算法.
  \begin{table}[!htbp]
%\caption{二分搜索算法的例子}
    \centering
%    \begin{tabular}{|l|l|l|l|l|l|l|l|l|l|l|l|l|}
\begin{tabular}{lccccccccccccc}
      下标 & 0 & 1 & 2 & 3 & 4 & 5 & 6 & 7 & 8 & 9 & 10 & 11 & 12\\
\hline
值  & 3  & 14 & 27 & 31 & 39 & 42 & 55 & 70 & 74 & 81 & 85 & 93 & 98\\
\hline
(1) & $l$ & & & & & & $m$ & & & & & & $r$\\
(2) &     & & & & & &     & $l$ & & $m$ & & & $r$ \\
(3)  &     & & & & & &     & $l,m$ & $r$ &  & & &    \\

      \hline 
      \end{tabular}
  \end{table}
(1): $m = (l+r)/2 = 6$, $x> A[m]$, $l = m+1$;  \\
(2): $m= (l+r)/2 = 9$, $x < A[m]$,  $r = m-1$;  \\
(3): $m = (l+r)/2 = 7$, $x=A[m]$, \textbf{return} $m$.
\end{frame}

\begin{frame}{2.4 大整数乘法}
研究高效的大整数乘法算法的具有广泛的现实需求
\begin{itemize}[<+-|alert@+>]
\item \textsf{int}: -32768-32767 范围的整数 
\item \textsf{unsigned int}: 0-65535 范围的正整数  
\item \textsf{long}: -2147483648-2147483647 的整数  
\item \textsf{unsigned long}: 0-4294967295 的正整数 
\end{itemize}
\end{frame}

\begin{frame}{笔算算法}
两个
$n$ 位整数相乘, 第一个数中的 $n$ 个数字都分别被第二个数中的 $n$ 个数字相乘, 这样就一共要做 $n^2$ 次乘法.  
\begin{table}
\centering
\begin{tabular}{c}
\qquad  2 3 \\
$\times$ \quad 1 4\\
\hline
\qquad 9 2\\
 2 3\\
\hline
\quad 3 2 2  
\end{tabular}
\end{table}
\end{frame}

\begin{frame}{整数相乘例子}
\textbf{例2.} 整数 $23$ 和 $14$ 相乘. 这两个数字可以这样表示:
\begin{exampleblock}{}  
 \begin{eqnarray*}
      23 & = & 2\times 10^1 + 3\times 10^0\\
     14 & = & 1\times 10^1 + 4\times 10^0 
 \end{eqnarray*}
\end{exampleblock}
现在把它们相乘: 
\begin{exampleblock}{}
   \begin{eqnarray*}
   23 \times 14 &  = & (2\times 10^1 + 3\times 10^0) \times (1\times 10^1 + 4\times 10^0)   \\
 & = & (2\times 1)10^2+(2\times 4+3\times 1)10^1+(3\times 4)10^0 
   \end{eqnarray*}
 \end{exampleblock}
 产生一个正确的结果 $322$, 但使用了 $4$ 次乘法.  
\end{frame}

\begin{frame}{整数相乘算法}
设 $X$ 和 $Y$ 都是 $n$ 位二进制数, 计算它们的乘积 $XY$.  将 $X$ 和 $Y$
都分为2段, 每段的长为 $n/2$ 位(假设 $n$ 是 2 的幂)
\begin{exampleblock}{}
  \begin{displaymath}
X=\underbrace{  \quad A \quad }_{n/2\ bit} \quad \underbrace{  \quad B \quad }_{n/2\ bit}\ ,  \quad Y=\underbrace{  \quad C \quad }_{n/2\ bit} \quad \underbrace{  \quad D \quad }_{n/2\ bit} 
\end{displaymath}
由此,  $X = A 2^{n/2} + B$, $Y = C 2^{n/2} + D$.   

例如, $x = 10110110$, $x_L = 1011$, $x_R = 0110$, 则   \\
$x = 1011\times 2^4 + 0110$. 
\end{exampleblock}
\end{frame}

\begin{frame}{}
  $X$ 和 $Y$ 的乘积为
  \begin{exampleblock}{}
    \begin{eqnarray*}
  XY & = & (A 2^{n/2} + B)(C 2^{n/2} + D)\\
& = & AC2^n+(AD+BC)2^{n/2}+BD
\end{eqnarray*}
计算 $XY$, 需要进行4次 $n/2$ 位整数的乘法,  以及3次整数加法, 还要做2次移位. 所有这些加法和移位共用 $O(n)$ 步运算  
\begin{eqnarray*}
  T(n) = \left\{\begin{array}{ll} 
O(1) & n = 1     \\
4T(n/2)+O(n) & n > 1  
\end{array} \right. 
\end{eqnarray*}
\end{exampleblock}
根据主定理得出 $T(n)=O(n^2)$.  
\end{frame}

\begin{frame}{高斯发现}
德国数学家 \emph{Carl Friedrich Gauss}(1777-1855) 很早就发现, 两个复数
乘法初看上去涉及 4 次乘法, 但实际上可以化简为 3 次乘法运算  
\begin{exampleblock}{}
  \begin{eqnarray*}
  (a+bi)(c+di) = ac - bd +(bc+ad)i   \\
bc + ad = (a+b)(c+d)-ac-bd. 
\end{eqnarray*}
\end{exampleblock}
\end{frame}

\begin{frame}{两数相乘 3 次乘法公式}
对于任何两位数 $a=a_1a_0$ 和 $b=b_1b_0$, 它们的积 $c$ 可以用这个公式来
计算:
\begin{exampleblock}{}
  \begin{displaymath}
  c=a\times b=c_210^2+c_110^1+c_0  
\end{displaymath}
其中, $c_2=a_1\times b_1$ 是第一个数字的积, $c_0=a_0\times b_0$ 是第二个数字的积, $c_1=(a_1+a_0)\times (b_1+b_0)  -(c_2+c_0)$ 是 $a$ 数字和与 $b$ 数字和的积减去 $c_2$ 与 $c_0$ 的和 
\end{exampleblock}
\end{frame}

\begin{frame}{$23 \times 14$ 的 3 次乘法运算}
  \begin{exampleblock}{}
     \begin{eqnarray*}
   23 \times 14 &  = & (2\times 10^1 + 3\times 10^0) \times (1\times 10^1 + 4\times 10^0)   \\
 & = & (2\times 1)10^2+(2\times 4+3\times 1)10^1+(3\times 4)10^0 
     \end{eqnarray*}
   \end{exampleblock}
   根据公式, $c_2 = a_1\times b_1=2\times 1$,   $c_0=a_0\times
   b_0=3\times 4$,    $c_1=(a_1+a_0)\times (b_1+b_0)-(c_2+c_0)$
   得出 
   
\begin{displaymath}
  2\times 4 + 3\times 1 = (2+3)\times (1+4)-(2\times 1)-(3\times 4) 
\end{displaymath}
\end{frame}

\begin{frame}{改进算法}
设 $X$ 和 $Y$ 都是 $n$ 位二进制数, 计算它们的乘积 $XY$. $X = A 2^{n/2} + B$, $Y = C 2^{n/2} + D$.
把 $XY$ 写成另一种形式

\begin{eqnarray*}
  XY=AC2^n+((A+B)(C+D)-AC-BD))2^{n/2}+BD  
\end{eqnarray*}
上式只需做 3 次 $n/2$ 位整数的乘法, 6 次加、减法和 2 次移位.
\end{frame}

\begin{frame}{Fun Time}
写出改进算法时间复杂性递归公式(做 3 次 $n/2$ 位整数的乘法, 6 次加、减法和 2 次移位
), 并用主定理计算 $T(n)$.
\begin{theorem}[主定理]
  \begin{eqnarray*}
    T(n) = \left\{
        \begin{array}{ll}
          \Theta(n^d) &\qquad  if\ a < b^d  \\
\Theta(n^d\log n) &\qquad  if \ a = b^d  \\
\Theta(n^{\log_ba}) &\qquad  if\ a > b^d
        \end{array}
\right. 
  \end{eqnarray*}
\end{theorem}
\end{frame}

\begin{frame}{Fun Time}
  \begin{exampleblock}{}
    \begin{eqnarray*}
  T(n)=\left\{\begin{array}{ll}
O(1) & n = 1   \\
3T(n/2)+O(n) & n>1   
\end{array} \right. 
    \end{eqnarray*}
  \end{exampleblock}
  其解为 $T(n)=O(n^{\log_2 3})=O(n^{1. 59})$.  
\end{frame}

\begin{frame}{2.5 Strassen 矩阵乘法}
\begin{itemize}[<+-|alert@+>]
\item 设 $A$ 和 $B$ 是两个 $n\times n$ 矩阵, 它们的乘积 $C$ 是一个
  $n\times n$ 矩阵. 乘积矩阵 $C$ 中元素 $c_{ij}$ 定义为
  
\begin{eqnarray*}
  c_{ij}=\sum_{k=1}^na_{ik}b_{kj} 
\end{eqnarray*}
\item 计算矩阵乘积 $C$, 每计算一个 $c_{ij}$, 需要做 $n$ 次乘法和 $n-1$ 次加法, 因此, 求出矩阵 $C$ 的 $n^2$ 个元素所需的时间为 $O(n^3)$ 
\end{itemize}
\end{frame}

\begin{frame}{Strassen 分治法}
\begin{itemize}[<+-|alert@+>]
\item 德国数学家 Volker Strassen 在
 1969 年提出采用分治方法, 将矩阵 $A$, $B$ 和 $C$ 中每一矩阵都分块成4个大小相等的子矩阵, 每个子矩阵都是 $n/2\times n/2$ 的方阵  

 可将方程 $C=AB$ 重写为:  
 \begin{exampleblock}{}   
\begin{eqnarray*}
  \left[\begin{array}{cc}
C_{11} & C_{12} \\
C_{21} & C_{22}
\end{array} \right]= \left[\begin{array}{cc}
A_{11} & A_{12} \\
A_{21} & A_{22}
\end{array} \right]  \left[\begin{array}{cc}
B_{11} & B_{12} \\
B_{21} & B_{22}
\end{array} \right] 
\end{eqnarray*}
\end{exampleblock}
\end{itemize}
\end{frame}

\begin{frame}{}
  由此可得
  \begin{exampleblock}{}
    \begin{eqnarray*}
  C_{11} & = & A_{11}B_{11}+A_{12}B_{21}\\
  C_{12} & = & A_{11}B_{12}+A_{12}B_{22}\\
  C_{21} & = & A_{21}B_{11}+A_{22}B_{21}\\
  C_{22} & = & A_{21}B_{12}+A_{22}B_{22} 
    \end{eqnarray*}
  \end{exampleblock}
  \begin{itemize}[<+-|alert@+>]
\item 如果 $n=2$, 则 2 个 2 阶方阵的乘积共需 8 次乘法和 4 次加法  
\item 当子矩阵的阶大于 2 时, 可以继续将子矩阵分块, 直到子矩阵的阶降为 2  
\end{itemize}
% 由此,产生了分治降阶的递归算法. 
\end{frame}

\begin{frame}{Strassen 矩阵乘积算法复杂性分析}
计算 2 个 $n$ 阶方阵的乘积转化为计算 8 个 $n/2$ 阶方阵的乘积和 4 个
$n/2$ 阶方阵的加法. 2 个 $n/2\times n/2$ 矩阵的加法可以在 $O(n^2)$ 时
间内完成.  
\begin{exampleblock}{}
  \begin{eqnarray*}
  T(n)=\left\{\begin{array}{ll}
O(1) & n=2   \\
8T(n/2)+O(n^2) & n > 2  
\end{array} \right. 
\end{eqnarray*}
\end{exampleblock}
该递归方程的解是 $T(n)=O(n^3)$. 
\end{frame}

\begin{frame}{Strassen 矩阵乘积改进算法}
                 %                  
  \begin{exampleblock}{Volker Strassen 算法}
\begin{eqnarray*}
  M_1 & = & A_{11}(B_{12}-B_{22})   \\
  M_2 & = & (A_{11}+A_{12})B_{22}  \\
  M_3 & = & (A_{21}+A_{22})B_{11}  \\
  M_4 & = & A_{22}(B_{21}-B_{11})   \\
  M_5 & = & (A_{11}+A_{22})(B_{11}+B_{22})   \\
  M_6 & = & (A_{12}-A_{22})(B_{21}+B_{22})  \\
  M_7 & = & (A_{11}-A_{21})(B_{11}+B_{12})  
\end{eqnarray*}
\end{exampleblock}
\end{frame}

\begin{frame}{Strassen 矩阵乘积改进算法}
  \begin{exampleblock}{}
    \begin{eqnarray*}
  C_{11} & = & M_5+M_4-M_2+M_6   \\
C_{12} & = & M_1+M_2   \\
C_{21} & = & M_3+M_4   \\
C_{22} & = & M_5+M_1-M_3-M_7   
\end{eqnarray*}
\end{exampleblock}
\end{frame}

\begin{frame}{Fun Time}
 Strassen 矩阵乘法中, 用了7次 $n/2$ 阶方阵乘积和18次 $n/2$ 阶方阵的加减法,
 分析该算法计算时间 $T(n)$:
 \begin{exampleblock}{}
     \begin{eqnarray*}
    T(n) = \left\{
        \begin{array}{ll}
          \Theta(n^d) &\qquad  if\ a < b^d  \\
\Theta(n^d\log n) &\qquad  if \ a = b^d  \\
\Theta(n^{\log_ba}) &\qquad  if\ a > b^d
        \end{array}
\right. 
  \end{eqnarray*}
\end{exampleblock}
\end{frame}

\begin{frame}{改进算法复杂性分析}
\begin{itemize}[<+-|alert@+>]
\item Strassen 矩阵乘法中, 用了7次 $n/2$ 阶方阵乘积和18次 $n/2$ 阶方阵
  的加减法, 该算法计算时间 $T(n)$:  
  \begin{exampleblock}{}
\begin{eqnarray*}
  T(n)= \left\{
    \begin{array}{ll}
       O(1) & n=2   \\
      7T(n/2)+O(n^2) & n>2        
    \end{array}\right. 
\end{eqnarray*}
\end{exampleblock}
\item 解此递归方程得 $T(n)=O(n^{\log_2 7})\approx O(n^{2. 81})$.  
\end{itemize}
\end{frame}

\begin{frame}{2.7 合并排序}
  \begin{exampleblock}{排序问题}
    给定长度为 $n$ 的一个数组, 设计一个有效的排序算法, 产生输入数组的
    一个重排, 使数组元素按从小到大的顺序排列.
  \end{exampleblock}
\end{frame}

\begin{frame}{合并排序算法基本思想}
\begin{itemize}[<+-|alert@+>]
\item[(1)]  将待排序数组 $A[0,n-1]$ 分成大小大致相同的 2 个子数组 $A[0,\lfloor n/2\rfloor-1]$ 和 $A[\lfloor n/2\rfloor,n-1]$   
\item [(2)] 分别对 2 个子数组进行排序   
\item [(3)] 将排好序的子数组合并成为一个有序数组   
\end{itemize}
\end{frame}

\begin{frame}{\textsf{MERGESORT} 算法}
\textbf{\textsc{MERGESORT}}$(A)$\\
1\quad \textbf{if} $len(A) <=1 $ \textbf{return} $A$ \\
2\quad  $q\gets len(A)/2$\\
3\quad  $L\gets A[0..q]$, $R\gets A[q+1..len(A)-1]$ \\
4\quad  $L \gets$ \textsf{MERGESORT}$(L)$\\
5\quad  $R \gets$ \textsf{MERGESORT}$(R)$\\
6\quad  \textbf{return} \textsf{MERGE}$(L, R)$\\
\end{frame}

\begin{frame}{合并排序操作过程}
\begin{figure}
  \centering
  \includegraphics[width=.8\textwidth]{./pictures/mergesort.png}%
%  \caption{合并操作序列}%
\end{figure}
\end{frame}

\begin{frame}{\textsf{MERGE} 算法}
\textbf{\textsc{MERGE}}($L, R$)\\
1\quad $i \gets 0, j \gets 0, aux \gets [\ ]$ \\
2\quad \textbf{while} $i < len(L)$ and $j < len(R)$\\
3\quad \quad \textbf{if} $L[i]\leq R[j]$\\
4\qquad \qquad  $aux.add(L[i])$ ; $i\gets i+1$\\
5\qquad  \textbf{else} $aux.add(R[j])$ ; $j\gets j+1$\\
6\qquad  \textbf{if} $i == len(L)$ \\
7\qquad  \qquad $aux.add(R[j:len(R)-1])$ \\
8\qquad  \textbf{else}  $aux.add(L[i:len(L)-1])$  \\
9\quad \textbf{return} $aux$
\end{frame}

\begin{frame}{Fun Time}
分析合并排序算法时间复杂性.
\begin{exampleblock}{}
  \begin{displaymath}
   T(n)=aT(\lceil n/b \rceil)+\Theta(n^d)
 \end{displaymath}
\end{exampleblock}
\begin{exampleblock}{}
  \begin{eqnarray*}
    T(n) = \left\{
        \begin{array}{ll}
          \Theta(n^d) &\qquad  \mathrm{if}\ a < b^d   \\
\Theta(n^d\log n) &\qquad  \mathrm{if} \ a = b^d   \\
\Theta(n^{\log_ba}) &\qquad  \mathrm{if}\ a > b^d  
        \end{array}
\right. 
  \end{eqnarray*}
\end{exampleblock}
\end{frame}

\begin{frame}{Fun Time}
\begin{itemize}[<+-|alert@+>]
%\item 算法 \textsf{MERGE} 每一次递归调用需要完成一次键值比较, 比较之后, 两个数组中尚需处理的元素总数个数减一, 因此, 可在 $O(n)$ 时间内完成, 
\item 合并排序算法最坏时间复杂性 $T(n)$ 满足 
  \begin{exampleblock}{}
    \begin{eqnarray*}
  T(n) = \left\{
    \begin{array}{ll}
      O(1) & n = 1   \\
2T(n/2)+O(n) & n > 1   
    \end{array}\right. 
\end{eqnarray*}
\end{exampleblock}
\item 根据主定理, 得出 $T(n)=O(n\log n)$.  
\end{itemize}
\end{frame}

\begin{frame}{2.8 快速排序}
\begin{itemize}[<+-|alert@+>]
\item 算法思想: 对给定数组中的元素进行重新排列, 确定数组中元素的一个位置 $q$, 得到一个快速排序的划分
%, 在一个划分中, 所有在 $s$ 下标之前的元素都小于等于 $A[s]$, 所有在
               %                $s$ 下标之后的元素都大于等于 $A[s]$ 
  \begin{exampleblock}{}
\begin{eqnarray*}
\underbrace{A[0]\cdots A[q]}_{ \leq A[q]} & A[q] & \underbrace{A[q+1]\cdots A[n-1]}_{\geq A[q]}  
\end{eqnarray*}
\end{exampleblock}
\item 建立一个划分之后, $A[q]$ 已经位于它在有序数组中的最终位置, 接下来需要继续对 $A[q]$ 前和 $A[q]$ 后的子数组分别进行快速排序 
\end{itemize}
\end{frame}

\begin{frame}{\textsf{QUICKSORT} 算法}
\textbf{QUICKSORT}$(A,p,r)$\\
1\qquad\   \textbf{if} $p < r$\\
2\qquad\ \textbf{then} $q\gets$  \textsf{PARTITION}$(A,p,r)$\\
3\qquad\ \qquad\ \textsf{QUICKSORT}$(A,p,q)$\\
4\qquad\ \qquad\  \textsf{QUICKSORT}$(A,q + 1,r)$
\end{frame}

\begin{frame}{快速排序算法的例子}
\begin{table}
%\footnotesize{
  \centering
  \begin{tabular}{lllllllll}
& 0 & 1 & 2 & 3 & 4 & 5 & 6 & 7 \\
\hline
(a) & {\textbf{5}} & $3^i$ & 1 & 9 & 8 & 2 & 4 & $7^j$    \\
(b) & {\textbf{5}} & 3 & 1 & \underline{$9^i$} & 8 & 2 & \underline{$4^j$} & 7   \\
(c) & {\textbf{5}} & 3 & 1 & $4^i$ & 8 & 2 & $9^j$ & 7   \\
(d) & {\textbf{5}} & 3 & 1 & 4 & \underline{$8^i$} & \underline{$2^j$} & 9 & 7   \\
(e) & {\textbf{5}} & 3 & 1 & 4 & $2^i$ & $8^j$ & 9 & 7   \\
(f) & {\textbf{5}} & 3 & 1 & 4 & \underline{$2^j$} & $8^i$ & 9 & 7   \\    
(g) & \underline{2} & 3 & 1 & 4 & \underline{{\textbf{5}}} &  8 & 9 & 7   \\
\\
%(h) & {\textbf{2}} & $3^i$ & 1 & $4^j$ & & & & \\
%(i) & {\textbf{2}} & $3^i$ & $1^j$ & 4 & & & & \\
%(j) & {\textbf{2}} & $1^i$ & $3^j$ & 4 & & & & \\
%(k) & {\textbf{2}} & $1^j$ & $3^i$ & 4 & & & & \\
%(l) & 1 & {\textbf{2}}  & 3 & 4 & & & & \\
%(m) & 1 &  &  &  & & & & \\
%(n) &  &  & {\textbf{3}}  & $4^{ij}$ & & & & \\
%(o) &  & &  {$\textbf{3}^j$}  &  $4^i$ & & & & \\
%(p) &  & &  & 4 & & & & \\
   \end{tabular}
 % \caption{快速排序算法的例子}
%}
\end{table}
\end{frame}

\begin{frame}{\textsf{PARTITION} 算法}
\textbf{PARTITION}$(A,p,r)$\\
1\qquad\  $x\gets  A[p]$, $i\gets  p+1$,   $j\gets  r$\\
2\qquad\  \textbf{while} $i \leq j$ \\
3\qquad\  \qquad \textbf{while} $A[j]\geq  x$ and $j > p$\\
4\qquad\  \qquad \qquad     $j\gets j-1$\\
5\qquad\ \qquad  \textbf{while}    $A[i]\le  x$ and $i < r$\\
6\qquad\  \qquad \qquad  $i\gets i+1$ \\
7\qquad\  \qquad       \textbf{if} $i < j$   \textbf{then}  $A[i]\leftrightarrow  A[j]$\\
8\qquad\ \qquad  \qquad $i\gets i+1$, $j\gets j-1$\\
9\qquad\   $A[p]\leftrightarrow A[j]$,  \textbf{return} $j$\\
\end{frame}

\begin{frame}{快速排序算法复杂性分析}
快速排序算法的运行时间依赖于:
\begin{itemize}[<+-|alert@+>]
\item 划分的平衡与否 
\item 划分的平衡与否依赖于算法的输入 
\item 如果划分平衡, 时间复杂性为 $\Theta(n\log n)$ 
\item 如果划分不平衡, 时间复杂性为 $\Theta(n^2)$  
\end{itemize}
\end{frame}

\begin{frame}{最坏时间复杂性}
 \textsc{Quicksort} 的最坏情况发生在 \textsc{Partition} 输出的两
  个区域中, 一个仅包含 1 个元素, 另一个包含 $n-1$ 个元素的情况;假设上
  述不平衡的划分发生在算法的每一步迭代中, 则
  \begin{exampleblock}{}
      \begin{displaymath}
    T(n)=\left\{\begin{array}{ll}
O(1) & n\leq 1   \\
T(n-1)+O(n) & n > 1   
\end{array}\right. 
\end{displaymath}
\end{exampleblock}
\end{frame}

% \begin{frame}{Fun Time}
% 解递归式:
% \begin{exampleblock}
%     \begin{displaymath}
%     T(n)=\left\{\begin{array}{ll}
% O(1) & n\leq 1   \\
% T(n-1)+O(n) & n > 1   
% \end{array}\right. 
% \end{displaymath}
% \end{exampleblock}
% \end{frame}
% %
% %\item 解递归式, 可以得出 $T(n) = O(n^2)$. 

% %注意: 算法的 $O(n^2)$ 运行时间发生在输入序列已经有序的情况. 

% \begin{frame}{最优时间复杂性}
%   \begin{itemize}[<+-|alert@+>]
%   \item 设如果\textsc{Partition} 算法产生两个大小为 $n/2$ 的区域, 则 
% \begin{displaymath}
%   T(n) = 2T(n/2) + O(n) 
% \end{displaymath}
% \item 根据主定理, 可以得出 
% \begin{displaymath}
%   T(n) = O(n\lg n).  
% \end{displaymath}
% \end{itemize}
% \end{frame}

\begin{frame}{随机化快速排序算法}
\begin{itemize}[<+-|alert@+>]
\item 快速排序算法取决于划分的对称性  
\item 采用随机策略进行划分  
\item 算法每一步在数组 $A[p,r]$ 中随机选出一个元素作为划分元素, 可以期望划分是较对称的 
\end{itemize}
\end{frame}

\begin{frame}{\textsf{RANDOMIZED-PARTITION}算法}

\textsf{RANDOMIZED-PARTITION}$(A,p,r)$\\
1\qquad  $i$=\textsf{Random}$(p,r)$  \\
2 \qquad exchange $A[p]\leftrightarrow A[i]$ \\
3 \qquad Return \textsf{PARTITION}$(A,p,r)$
\end{frame}

\begin{frame}{\textsf{RANDOMIZED-QUICKSORT}算法}
\textsf{ RANDOMIZED-QUICKSORT}$(A,p,r)$\\
1\qquad \textbf{if} $p< r$\\
2\qquad  \textbf{then} $q=$\textsf{RANDOMIZED-PATITION}$(A,p,r)$\\
3\qquad \quad \textsf{RANDOMIZED-QUICKSORT}$(A,p,q)$\\
4\qquad \quad  \textsf{RANDOMIZED-QUICKSORT}$(A,q+1,r)$
\end{frame}

\begin{frame}{本章小结}
 \begin{itemize}[<+-|alert@+>]
 \item 分治法是一种通用算法设计技术
 \item 许多分治算法的时间复杂性 $T(n)$ 满足方程 
   \begin{displaymath}
     T(n)=aT(n/b)+f(n)
   \end{displaymath}
 \item (Master)主定理确定了该方程解的增长次数
 \end{itemize}
\end{frame}
\end{document}

%%% Local Variables:
%%% mode: latex
%%% TeX-master: t
%%% End:
